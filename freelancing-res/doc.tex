\documentclass{article}
\usepackage{graphicx} % Required for inserting images

\title{Freelancing}
\author{Riccardo Parrino}
\date{May 2025}

\begin{document}

\maketitle

\section{Aspetti tecnici}
\subsection{Organizzazione del lavoro}
\subsection{Stack Tecnologico}

\section{Aspetti legali}

\subsection{Come scrivere un contratto}
\subsection{Partita IVA}
\subsubsection{Apertura partita iva}
Per aprire partita iva \'e necessario inviare il modello AA9/12 o telematicamente o fisicamente all' Agenzia delle entrate. 

All'interno di questo modello \'e necessario inserire il proprio nome e cognome se si tratta di lavoratore autonomo, della denominazione dell'azienda se si tratta di un'impresa. Successivamente \'e necessario inserire il codice ATECO della propria attivit\'a, il tipo di regime fiscale (ordinario, ordinario semplificato o forfettario, supponendo che si soddisfino tutti i requisiti), il luogo in cui si svolge l'attivit\'a ed eventuali collaboratori come anche il prorpio commercialista.

Per quanto riguarda il luogo di svolgimento della propria attivit\'a \'e necessario indicare un luogo in cui si \'e facilmente reperibili ed \'e apposta eventuale targhetta di riconoscimento. Se si \'e in affitto, \'e necessario informare il proprietario di casa che si intende usare l'abitazione come ufficio.

Oltre a questi adempimenti in talune situazioni \'e necessario iscriversi all'INPS per gestione separata o artigiani/commercianti, all'INAIL in caso di rischio per la salute, alla Camera di Commercio (per artigiani e commercianti) e SCIA al Comune in caso di  attivit\'a soggetta a comunicazione di inizio.

Successivamente a questi passaggi \'e necessario avere una propria PEC (Posta Elettronica Certificata) e se non si rientra nel regime forfettario \'e obbligatoria la fatturazione elettronica tramite SDI. 

In genere i tempi di apertura sono immediati ed i costi sono ridotti ai soli servizi professionali (come il commercialista), i contributi INPS ed eventuali diritti camerali.

\subsubsection{Chiusura partita iva}
Inviare il modello AA9/12 di chiusura, in maniera analoga a quanto fatto per l'apertura. Non ci devono pi\'u essere fatture da emettere. Per almeno un anno, bisogno mantenere i rapporti con il proprio consulente, per la dichiarazione dei redditi e eventuali F24 da pagare per l'anno precedente.
\subsubsection{Codice ateco}
Il codice ATECO (acronimo di ATtivit\'a ECOnomiche) \'e una classificazione alfanumerica usata in Italia per identificare il tipo di attivit\'a economica svolta da un'impresa o da un lavoratore autonomo. Il codice ATECO \'e fondamentale per l'apertura della partita IVA, la determinazione del regime fiscale, contributivo e per fini statistici. 

Il codice ATECO ha una struttura gerarchica XX.YY.ZZ.W, in cui XX individua la sezione, YY individua la divisione, ZZ il gruppo e W la Classe/Voce, anche se quest'ultimo non \'e sempre presente e serve a dare un ulteriore livello di dettaglio. Un esempio pratico: Codice ATECO: 74.10.21 significa 74 Altre attivit\'a professionali, scientifiche e tecniche, 10 si intende Attivit\'a di design specializzate e 21 Design grafico, in particolare per il web. Il codice ATECO si usa per l'apertura della partita IVA, per il calcolo dei contributi INPS, per la scelta del regime forfettario,, per statistiche ISTAT ed in alcuni casi per autorizzazione e licenze.

Per attivit\'a legate allo sviluppo software o legate all'informatica, la divisione di riferimento sono la 62 e la 63. 
Per una spiegazione dettagliata e precisa, fare ricerche al seguente link: https://www.istat.it/classificazione/classificazione-delle-attivita-economiche-ateco


\subsubsection{Regime fiscale}
\subsubsection{Cassa previdenziale}
\subsubsection{Gestione Artigiani e Commercianti INPS}

\subsubsection{Cartelle esattoriali}
Le cartelle esattoriali sono atti ufficiali di riscossione meessa dall'agenzia delle Entrate per richiesere il pagamento di debiti verso lo Stato o altri enti pubblici. In genere contengono: l'importo da pagare (comprensivo di debito originario, sanzioni, interessi di mora, eventuali spese di notifica e riscossione), l'ente creditore (es. INPS, Comune, Agenzia delle Entrate, ...), le istruzioni per il pagamento, il termine per pagare o fare ricorso. 

Esistono diversi motivi per cui si potrebbero ricevere delle cartelle esattoriali, come ad esempio: Omissioni o ritardi nei versamenti fiscali (IRPED, IVA, imposta sostitutiva forfettaria), Contribuit INPS non versati, Sanzioni per errori fiscali, multe per mancata comunicazione (esterometro, CU omesse, etc.), debiti pregressi di cui si \'e a conoscenza. Se non si paga una determinata cartella esattoriale si rischia un fermo amministrativo su un auto, il pignoramento di un conto corrente o dello stipendio, o un'ipoteca su immobili. 

Ricevendo una cartella, \'e bene verificare che sia corretta, controllando che il debito sia erale e verificare che non sia caduto in prescrizione, si pu\'o rateizzare il pagamento o si pu\'o fare ricorso entro 60 giorni se la si ritiene errata. 

\subsubsection{Contenziosi: Agenzia delle Entrate}
\begin{itemize}
    \item Omissioni o errori nella fatturazione
    \item Compensi non dichiarati
    \item Superamento soglie del regime forfettario, ma non si passa al regime ordinario
    \item Deducibilità di spese non inerenti (viaggi, cene, prodotti ad uso personale non legate all'attivit\'a produttiva)
    \item Mancata iscrizione alla gestione previdenziale INPS (o errata)
    \item Errori nei versamenti di imposte e contributi
    \item Errori nei rapporti con clienti esteri (mancata iscrizione al VIES, reverse charge non applicato)
\end{itemize}

\subsubsection{Contenziosi: Clienti}
\begin{itemize}
    \item Mancato pagamento o ritardo nei pagamenti
    \item Ambiguità o assenza di contratto
    \item Scope creep (espansione non concordata delle richieste)
    \item Diritti di propriet\'a intellettuale
    \item Violazione della privacy o del GDPR
    \item Contenziosi con agenzie o intermediari
    \item Conflitti sulla qualit\'a del software
\end{itemize}

\subsection{Tributi}

\subsubsection{INPS}
L’INPS gestisce la liquidazione e il pagamento delle pensioni e delle indennità di natura previdenziale e assistenziale. Le pensioni sono prestazioni previdenziali, determinate sulla base di rapporti assicurativi e finanziate con i contributi di lavoratori e aziende pubbliche e private. Invece, le prestazioni assistenziali o “a sostegno del reddito” tutelano i lavoratori che si trovano in particolari momenti di difficoltà della loro vita lavorativa e provvedono al pagamento di somme destinate a coloro che hanno redditi modesti e famiglie numerose. Per alcune di queste prestazioni l’INPS è coinvolto solo nella fase di erogazione, mentre per altre svolge tutto il procedimento di assegnazione (fonte: sito INPS). L’INPS nasce oltre centoventicinque anni fa, nel 1898 con la fondazione della Cassa Nazionale di previdenza per l'invalidità e per la vecchiaia degli operai, allo scopo di garantire i lavoratori dai rischi di invalidità, vecchiaia e morte. Con il tempo l’Istituto ha assunto un ruolo di crescente importanza fino a diventare il pilastro del sistema nazionale del welfare.

\subsubsection{Modello F24}
Il modello F24 \'e un modulo utilizzato per il pagamento di tasse, imposte, contributi e tributi e serve a versare importi dovuti all'Agenzia delle Entrate, all'INPS, ai Comuni o ad altri enti. 

Attraverso il modello F24 si possono pagare: imposte sui redditi (IRPEF, IRES), IVA, IMU, TASI, TARI, Contributi INPS e INAUL, Sanzioni e Interessi, Accise e tributi regionali e locali

Esistono diverse versioni del modello F24: F24 ordinario, F24 semplificato, F24 accise ed F24 ELIDE.

Il modello F24 pu\'o essere pagato Online tramite i servizi telematici dell'Agenzia delle Entrate o del proprio home banking, in banca o posta o tramite intermediari abilitati come i commercialisti.

\section{Aspetti economici}
\subsection{Modi di guadagnare}
\subsection{Quanto farsi pagare}

\section{Aspetti marketing}
\subsection{Trovare clienti}
\subsection{Personal Branding}

\section{Aspetti relazionali}
\subsection{Gestione clienti}

\end{document}
