\documentclass{article}
\usepackage{graphicx} % Required for inserting images

\title{Freelancing}
\author{Riccardo Parrino}
\date{May 2025}

\begin{document}

\maketitle

\section{Aspetti tecnici}
\subsection{Organizzazione del lavoro}
\subsection{Stack Tecnologico}

\section{Aspetti legali}

\subsection{Fatturazione}

\subsubsection{Testo di riferimento}
Il testo di referimento per qualsiasi dubbio o problema \'e il "Testo Unico IVA".

\subsubsection{Come emettere fattura}
I dati necessari per una corretta emissione della fattura sono: partita iva destinatario, pec destinatario, codice fiscale, email, 
dettagli del prodotto o servizi, parametri fiscali (profilo fiscale) condizioni e informaizoni di pagamento, ed eventuali coordinate bancarie alla quale inviare il saldo.


\subsubsection{Quando emettere fattura}
Le principali situazioni in cui emettere fattura sono la vendita di beni e/o la prestazione di servizi. 

Per quanto riguarda la vendita di bene, la fattura deve essere emessa entro 12 giorni dalla data di effettuazione dell'operazione (quindi alla consegna o alla spedizione del bene), trattandosi quindi di fattura immediata,
oppure entro il 15 del mese successivo, se si fa riferimento a documenti di trasporto (DDT) o altri documenti che provano la consegna.

Nel caso di prestazioni di servizio, se la fattura \'e immediata, questa va emessa entro 12 giorni dalla data in cui \'e stata effettuata la prestazione.
Se il pagamento avviene in anticipo, la fattura va emessa alla data del pagamento anche solo parziale. 

Per i soggetti obbligati (la maggior parte delle partite IVA), la fattura deve essere inviata al Sistema di Interscambio (SdI) entro i termini sopraindicati. La data di emissione coincide con la data di trasmissione allo SdI.

Se si dovesse emettere una fattura in ritardo, si rischiano sanzioni amministrative che possono andare dal 90\% al 180\% dell'IVA non documentata correttamente. 

\subsubsection{Sistema di Interscambio}
Per Sistema di Interscambio si intende il canale telematico dell'Agenzia delle Entrate per la gestione della fatturazione eletteonica.

Il Sistema di Interscambio riceve le fatture elettroniche, effettua i controlli su questa e le invia ai destinatari. 

Il Sistema di Interscambio verifica che siano presenti le informazioni minime obbligatorie previste per legge 
(estremi identificativi del fornitore e del cliente, il numero, la data della fattura, la descrizione della natura, la quantit\'a, la qualit\'a del bene ceduto o del servizio perstato, l'imponibile, l'aliquota e l'IVA),
che i valori della partita IVA del fornitore siano esistenti, e quindi presenti in Anagrafe Tributaria, che sia compilato il campo "Codice Destinatario" e che ci sia coerenza tra i valori dell'imponibile, dell'aliquota e dell'IVA.

Se fosse presente un qualche tipo di errore, il SdI invia una notifica di scarto e la fattura viene considerata non emessa. 

Se la fattura \'e considerata corretta allora viene trasmessa all'indirizzo riportato nel campo "Codice Destinatario" e "PEC Destinatario" del file.

\subsection{Come scrivere un contratto}
\subsection{Partita IVA}
\subsubsection{Apertura partita iva}
Per aprire partita iva \'e necessario inviare il modello AA9/12 o telematicamente o fisicamente all' Agenzia delle entrate. 

All'interno di questo modello \'e necessario inserire il proprio nome e cognome se si tratta di lavoratore autonomo, della denominazione dell'azienda se si tratta di un'impresa. Successivamente \'e necessario inserire il codice ATECO della propria attivit\'a, il tipo di regime fiscale (ordinario, ordinario semplificato o forfettario, supponendo che si soddisfino tutti i requisiti), il luogo in cui si svolge l'attivit\'a ed eventuali collaboratori come anche il prorpio commercialista.

Per quanto riguarda il luogo di svolgimento della propria attivit\'a \'e necessario indicare un luogo in cui si \'e facilmente reperibili ed \'e apposta eventuale targhetta di riconoscimento. Se si \'e in affitto, \'e necessario informare il proprietario di casa che si intende usare l'abitazione come ufficio.

Oltre a questi adempimenti in talune situazioni \'e necessario iscriversi all'INPS per gestione separata o artigiani/commercianti, all'INAIL in caso di rischio per la salute, alla Camera di Commercio (per artigiani e commercianti) e SCIA al Comune in caso di  attivit\'a soggetta a comunicazione di inizio.

Successivamente a questi passaggi \'e necessario avere una propria PEC (Posta Elettronica Certificata) e se non si rientra nel regime forfettario \'e obbligatoria la fatturazione elettronica tramite SDI. 

In genere i tempi di apertura sono immediati ed i costi sono ridotti ai soli servizi professionali (come il commercialista), i contributi INPS ed eventuali diritti camerali.

\subsubsection{Chiusura partita iva}
Inviare il modello AA9/12 di chiusura, in maniera analoga a quanto fatto per l'apertura. Non ci devono pi\'u essere fatture da emettere. Per almeno un anno, bisogno mantenere i rapporti con il proprio consulente, per la dichiarazione dei redditi e eventuali F24 da pagare per l'anno precedente.
\subsubsection{Codice ateco: Concluso}
Il codice ATECO (acronimo di ATtivit\'a ECOnomiche) \'e una classificazione alfanumerica usata in Italia per identificare il tipo di attivit\'a economica svolta da un'impresa o da un lavoratore autonomo. Il codice ATECO \'e fondamentale per l'apertura della partita IVA, la determinazione del regime fiscale, contributivo e per fini statistici. 

Il codice ATECO ha una struttura gerarchica XX.YY.ZZ.W, in cui XX individua la sezione, YY individua la divisione, ZZ il gruppo e W la Classe/Voce, anche se quest'ultimo non \'e sempre presente e serve a dare un ulteriore livello di dettaglio. Un esempio pratico: Codice ATECO: 74.10.21 significa 74 Altre attivit\'a professionali, scientifiche e tecniche, 10 si intende Attivit\'a di design specializzate e 21 Design grafico, in particolare per il web. Il codice ATECO si usa per l'apertura della partita IVA, per il calcolo dei contributi INPS, per la scelta del regime forfettario,, per statistiche ISTAT ed in alcuni casi per autorizzazione e licenze.

Per attivit\'a legate allo sviluppo software o legate all'informatica, la divisione di riferimento sono la 62 e la 63. 
Per una spiegazione dettagliata e precisa, fare ricerche al seguente link: https://www.istat.it/classificazione/classificazione-delle-attivita-economiche-ateco

\subsubsection{Il commercialista}
Cosa fa: si occupa della tenuta della contabilit\'a, 
predisposizione del bilancio di esercizio e della dichiarazione dei redditi, 
redazione dei libri contabili e libro unico del lavoro, 
svolgimento di perizie tecniche, ispezioni e revisioni amministrative.

La professione del commercialista pu\'o essere definita come quella del consulente che opera 
nel campo del diritto commerciale, del diritto tributario, della ragioneria e della contabilit\'a in generale.

Il diritto commerciale \'e una branca del diritto privato che regola i rapporti attinenti alla produzione e allo scambio della ricchezza. Pi\'u in particolare, regola ed ha per oggetto i contratti conclusi tra operatori economici e tra essi ed i loro i clienti privati ed anche gli atti e le attivit\'a delle societ\'a. 

Il diritto tributario \'e un settore del diritto finanziario che regolamente i tributi di ogni tipo.

La ragioneria \'e una disciplina che si occupa della rilevazinoe dei fenomeni aziendali e della loro traduzione in scritture cnotabili, allo scopo di fornire alle direzioni delle aziende 
o ad altri soggetti interessati le informazioni necessarie per esercitare le funzioni di controllo e di decisione.

Per contabilit\'a si intende al storia economica di un'azienda nel senso che conserva una traccia di tutte le operazioni commerciali.

\subsection{Regime fiscale}

\subsubsection{Regime fiscale forfetario}
Come riportato sul sito dell'agenzia delle entrate, "il regime forfetario prevede rilevanti semplificazioni ai fini IVA e ai fini contabili, e consente, altresì, la determinazione forfetaria del reddito da assoggettare a un’unica imposta in sostituzione di quelle ordinariamente previste, nonché di accedere ad un regime contributivo opzionale per le imprese."
Inoltre, \'e possibile accedere al regime forfetario senza vincoli di tempo ne di et\'a, ma soltanto soddisfando i requisiti previsti.
Il regime forfetario \'e stato introdotto dalla legge di stabilit\'a 2015 e rappresenta il regime naturale delle persone fisiche che esercitano un'attivit\'a di impresa, arte o professione in forma individuale.

Per quanto riguarda reddito e tassazione, per chi aderisce al regime forfetario \'e prevista un'unica imposta sostitutiva delle imposte sui redditi, delle addizionali regionali e comunali e dell'IRAP, nella misura del 15\%. 
L'aliquota viene applicata sull'imponibile determinato dall'ammontare dei ricavi o dei compensi percepiti moltiplicato per il coefficiente di redditivit\'a, diversificato a seconda del codice ATECO che contraddistingue l'attivit\'a esercitata.

Per i codice ATECO 62-63 \'e previsto un coefficiente di redditivit\'a del 67\%.

Per chi aderisce al regime forfetario, per i primi cinque anni di attivit\'a si potrebbe avvalere anche di un ulteriore detassazione, un'imposta sostitutiva del 5\%, visti alcuni requisiti da soddisfare.

Il regime forfetario \'e di pi\'u semplice gestione rispetto al regime ordinario od ordinario semplificato, sia per fini IVA, sia per fini sulle imposte dirette. 

Per quanto riguarda i fini Iva, non deve essere addebitata l'Iva in fattura e non deve essere detratta l'iva sugli acquisti. Non si liquida l'imposta, non si versa e non si \'e obbligati a presentare la dichiarazione e la comunicazione annuale Iva. 
Non \'e necessario comunicare all'Ageniza delle entrate le operazioni rilevanti ai fini Iva (ovvero lo spesometro) né quelle effettuate nei confronti di operatori economici aventi sede, residenza o domicilio in Paesi cosiddetti black list. Chi applica il regime forfetario, inoltre, non ha l'obbligo di registrare i corrispettivi, le fatture emesse e le ricevute.

Le semplificazioni ai fini delle imposte sul reddito riguardano l'esonero dagli obblighi di registrazione e tenuta delle scritture contabili, ma mantenere e cnoservare i registri previsti da disposizioni diverse da quelle tributarie.
Non operano le ritenute alla fonte, non si subiscono le ritenute, non si applicano le ritenute alla fonte. 

Ai fini Iva, si ha l'obbligo di numerare e conservare le fatture di acquisto e le bollette doganali, certificare i corrispettivi, 
integrare le fatture per le operazioni di cui risultatno debitori di imposta con l'indicazinoe dell'aliquota e della relativa imposta, da versare entro il giorno 16 del mese successivo a quello di effettuazione delle operazioni, senza diritto alla detrazione dell'imposta relativa.

La legge di stabilit\'a, insieme alla legge di bilancio, costituisce la manovra di finanza pubblica per il triennio di riferimento
e rappresenta lo strumento principale di attuazione degli obiettivi programmatici definiti con la Decisione di finanza pubblica. (fonte: "https://www.rgs.mef.gov.it/VERSIONE-I/attivita_istituzionali/formazione_e_gestione_del_bilancio/bilancio_di_previsione/bilancio_finanziario/BF-2016_2018/LdS2016/index.html#:~:text=La\%20legge\%20di\%20stabilit\%C3\%A0\%2C\%20insieme,la\%20Decisione\%20di\%20finanza\%20pubblica.")

\subsubsection{Scadenze regime fiscale forfetario}

\begin{comment}
    tema imposte
    tema contributi
\end{comment}

I principali adempimenti di chi aderisce al regime forfetario sono il versamento delle imposte, il versamento dei contributi previdenziali, il versamento del diritto comerale annuale, 
invio delle fatture elettroniche e la dichiarazione dei redditi.

Per quanto riguarda il versamento delle imposte, ci sono due principali scadenze: il 30 giugno ed il 30 novembre.
Il 30 giugno si versa il saldo dell'imposta sostitutiva riferita all'anno precedente e il 50\% dell'acconto dell'imposta sostitutiva dell'anno in corso. I versamenti del 30 giugno possono essere rateizzati fino a 6 rati, con cadenza mensile.

L'altra scadenza \'e invece quella del 30 novembre, procedendo al versamento del restante 50\% dell'acconto per le imposte dell'anno in corso. Quest'ultimo versamento non pu\'o essere rateizzato.

Per chi apre partita IVA, durante il primo anno non dovr\'a pagare alcuna imposta sostitutiva durante il primo anno, 
ma durante il secondo anno verser\'a l'ammontare del primo anno e l'acconto del secondo anno.

I contribuenti forfetari pagano un'imposta sostitutiva di IRPEF e addizionali regionali/comunali pari al 15\%, su un imponibile determinato sulla base del totale dei ricavi e dei compensi incassati (vige il principio di cassa) applicando la percentuale di redditivit\'a,
determinata sulla base del codice ateco, e deducendone i contributi previdenziali versati nel periodo.

Oltre al versamento dell'imposta sostitutiva \'e prevista anche l'imposta di bollo sulle fatture elettroniche, ovvero:
per ogni fattura emessa con importo superiore a 77,47 euro, per la cifra di 2 euro per ciascuna fattura. Le scadenze previste sono le seguenti:
per le fatture emesse nel primo, secondo e terzo trimestre, entro il 30 novembre, mentre per le fatture trasmesse nel quarto trimestre (quindi, ottobre-dicembre) entro il 28 febbraio. 

Oltre alle imposte, il soggetto a partita IVA forfetaria \'e tenuto al versamento dei contributi previdenziali.
L'aspetto contributivo dipende dall'attivit\'a svolta dal contribuente.
L'ammontare dovuto ed i termini in cui pagarlo \'e determinato dall'attivit\'a del contribuente.

Innanzi tutto bisogna distinguere in contribuenti che hanno una propria cassa previdenziale ed in coloro che non ce l'hanno e devono quindi affidarsi alla Gestione Separata INPS.
Ed inoltre, tra coloro che non hanno una propria cassa previdenziali, gli artigiani ed i commercianti dai restanti contribuenti. 

Per chi ha una propria cassa previdenziale, come gli avvocati, gli ingegneri e gli architetti ecc., le modalit\'a e le scadenze sono definite dalla cassa previdenziale stessa. 

Per chi invece non ha una propria cassa previdenziale deve iscriversi alla Gestione Separa INPS, pagando ogni anno un versamento in misura percentuale sul reddito forfetizzato di ogni anno.
Le aliquote variano tra il 26,07\% per chi non ha alcuna altra copertura previdenziale ed il 24,00\% per chi \'e pensionato o \'e gi\'a iscritto ad un'altra forma di previdenza obbligatoria.

Le scadenze previste per questi versamenti sono le stesse per quelle previste dall'imposta sostitutiva, 30 giugno e 30 novembre.

Come si diceva, vi \'e un'ulteriore distinsione per Artigiani e Commercianti, secondo la quale \'e prevista un'imposta di reddito detta "minimale" di 18.415 euro, da versare entro l'anno, indipendentemente dal reddito prodotto.
In altre parole, anche se in un determinato anno non si produce alcune reddito, si \'e comunque tenuti a pagare alla Gestione separata INPS, un contributo di 18.415 euro.

Se il proprio reddito, invece, supera l'importo minimale, allora andranno versati ulteriori contributi pari al 24\% (da verificare ogni anno) sul reddito eccedente.

Le scadenze sono articolate in quattro date, ovvero: 16 maggio, 20 agosto, 16 novembre, 16 febbraio (dell'anno successivo).

Per le attivit\'a che devono essere registrate alla camera di commercio (come gli artigiani ed i commercianti) \'e previsto anche il versamento del diritto camerale annuale alla Camera di Commercio. L'importo oscilla tra 44 e 53 euro. L'import deve essere versato entro il 30 giugno di ogni anno.

Dal primo gennaio 2024 anche i contribuenti in regime forfetario devono emettere fatture elettroniche, per beni (al momento della consegna o spedizione, incasso del corrispettivo se precedente la consegna o la spedizione)
e servizi (al momento dell'incasso del corrispettivo). In caso di fatturazione differita di beni o servizi, la fattura deve essere emessa e trasmessa entro il giorno 15 del mese successivo.

Infine, per chi applica il regime forfettario \'e prevista la repsentazione della dischiarazione dei redditi annuale, prevista entro il 30 settembre di ogni anno. 

\subsubsection{Come si pagano le imposte ed i contributi}
Sia le imposte che i contributi possono essere pagate tramite modello F24. 
Il modello F24 pu\'o essere pagato online tramite home banking, tramite commercialista o tramite banca o posta.

In alternativa al modello F24 esistono comunque diversi servizi telematici come FiscoOnline ed Entratel. Spesso comunque, affidandosi ad un commercialista, questo si occupa della trasmissione e del pagamento di contributi e imposte.

\subsubsection{Regime fiscale ordinario}
\subsubsection{Regime fiscale ordinario semplificato}

\subsubsection{Cassa previdenziale}
\subsubsection{Gestione Artigiani e Commercianti INPS}

\subsubsection{Cartelle esattoriali}
Le cartelle esattoriali sono atti ufficiali di riscossione meessa dall'agenzia delle Entrate per richiesere il pagamento di debiti verso lo Stato o altri enti pubblici. In genere contengono: l'importo da pagare (comprensivo di debito originario, sanzioni, interessi di mora, eventuali spese di notifica e riscossione), l'ente creditore (es. INPS, Comune, Agenzia delle Entrate, ...), le istruzioni per il pagamento, il termine per pagare o fare ricorso. 

Esistono diversi motivi per cui si potrebbero ricevere delle cartelle esattoriali, come ad esempio: Omissioni o ritardi nei versamenti fiscali (IRPED, IVA, imposta sostitutiva forfettaria), Contribuit INPS non versati, Sanzioni per errori fiscali, multe per mancata comunicazione (esterometro, CU omesse, etc.), debiti pregressi di cui si \'e a conoscenza. Se non si paga una determinata cartella esattoriale si rischia un fermo amministrativo su un auto, il pignoramento di un conto corrente o dello stipendio, o un'ipoteca su immobili. 

Ricevendo una cartella, \'e bene verificare che sia corretta, controllando che il debito sia erale e verificare che non sia caduto in prescrizione, si pu\'o rateizzare il pagamento o si pu\'o fare ricorso entro 60 giorni se la si ritiene errata. 

\subsubsection{Contenziosi: Agenzia delle Entrate}
\begin{itemize}
    \item Omissioni o errori nella fatturazione
    \item Compensi non dichiarati
    \item Superamento soglie del regime forfettario, ma non si passa al regime ordinario
    \item Deducibilità di spese non inerenti (viaggi, cene, prodotti ad uso personale non legate all'attivit\'a produttiva)
    \item Mancata iscrizione alla gestione previdenziale INPS (o errata)
    \item Errori nei versamenti di imposte e contributi
    \item Errori nei rapporti con clienti esteri (mancata iscrizione al VIES, reverse charge non applicato)
\end{itemize}

\subsubsection{Contenziosi: Clienti}
\begin{itemize}
    \item Mancato pagamento o ritardo nei pagamenti
    \item Ambiguità o assenza di contratto
    \item Scope creep (espansione non concordata delle richieste)
    \item Diritti di propriet\'a intellettuale
    \item Violazione della privacy o del GDPR
    \item Contenziosi con agenzie o intermediari
    \item Conflitti sulla qualit\'a del software
\end{itemize}

\subsection{Tributi}

\subsubsection{INPS}
L’INPS gestisce la liquidazione e il pagamento delle pensioni e delle indennità di natura previdenziale e assistenziale. Le pensioni sono prestazioni previdenziali, determinate sulla base di rapporti assicurativi e finanziate con i contributi di lavoratori e aziende pubbliche e private. Invece, le prestazioni assistenziali o “a sostegno del reddito” tutelano i lavoratori che si trovano in particolari momenti di difficoltà della loro vita lavorativa e provvedono al pagamento di somme destinate a coloro che hanno redditi modesti e famiglie numerose. Per alcune di queste prestazioni l’INPS è coinvolto solo nella fase di erogazione, mentre per altre svolge tutto il procedimento di assegnazione (fonte: sito INPS). L’INPS nasce oltre centoventicinque anni fa, nel 1898 con la fondazione della Cassa Nazionale di previdenza per l'invalidità e per la vecchiaia degli operai, allo scopo di garantire i lavoratori dai rischi di invalidità, vecchiaia e morte. Con il tempo l’Istituto ha assunto un ruolo di crescente importanza fino a diventare il pilastro del sistema nazionale del welfare.

\subsubsection{INAIL}
L'INAIL (Istituto Nazionale per l'Assicurazione contro gli Infortuni sul Lavoro) \'e un ente pubblico italiano che si occupa di assicurazione contro gli infortuni sul lavoro.
In generale offre servizi per la tutela dei lavoratori in caso di danni alla salute causati dallo svolgimento dell'attivit\'a lavorativa, promuove iniziative per migliorare la sicurezza nei luoghi di lavoro, 
fornisce supporto medico, riabilitativo ed anche professionale ai lavoratori infortunati o malati ed eroaga compensazioni economiche in caso di infortunio o malattia professinoale che comporti inabilit\'a temporanea, permanente o, nei casi pi\'u gravi, la morte del lavoratore.

Per alcune attivit\'a considerate rischiose (artigianato, edilizia, industria, ecc.), l'assicurazione INAIL \'e obbligatoria (da parte dei datori di lavoro).

\subsubsection{Modello F24}
Il modello F24 \'e un modulo utilizzato per il pagamento di tasse, imposte, contributi e tributi e serve a versare importi dovuti all'Agenzia delle Entrate, all'INPS, ai Comuni o ad altri enti. 

Attraverso il modello F24 si possono pagare: imposte sui redditi (IRPEF, IRES), IVA, IMU, TASI, TARI, Contributi INPS e INAUL, Sanzioni e Interessi, Accise e tributi regionali e locali

Esistono diverse versioni del modello F24: F24 ordinario, F24 semplificato, F24 accise ed F24 ELIDE.

Il modello F24 pu\'o essere pagato Online tramite i servizi telematici dell'Agenzia delle Entrate o del proprio home banking, in banca o posta o tramite intermediari abilitati come i commercialisti.

\section{Aspetti economici}
\subsection{Modi di guadagnare}
\subsection{Quanto farsi pagare}

\section{Aspetti marketing}
\subsection{Trovare clienti}
\subsection{Personal Branding}

\section{Aspetti relazionali}
\subsection{Gestione clienti}

\end{document}
